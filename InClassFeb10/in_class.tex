\documentclass{article}
\usepackage{amsmath, amssymb, cancel, mathtools, bm}
\usepackage[left=.5in, right=.5in, top=1in, bottom=1in]{geometry}
\usepackage[most]{tcolorbox}
\usepackage[skip=1em,indent=0pt]{parskip}
\setlength{\parindent}{0pt}
\newcommand{\statvec}[1]{\underset{\sim}{\bm{#1}}} % Vector symbol (tilde under vector)
\DeclareMathOperator{\EX}{\mathbb{E}} % Expected Value symbol
\newcommand{\indep}{\perp\!\!\!\!\perp} % Independence symbol
\newcommand{\real}{\mathbb{R}} % Simplified 'Reals' indicator
\newcommand{\pto}{\overset{P}{\to}}
\newcommand{\asto}{\overset{a.s.}{\to}}
\newcommand{\rthto}{\overset{L^r}{\to}}
\newcommand{\dto}{\overset{D}{\to}}
% Force all aggregate symbols to always put values above/below
\let\Oldint=\int
\let\Oldsum=\sum
\let\Oldprod=\prod
\let\Oldbigcup=\bigcup
\let\Oldbigcap=\bigcap
\let\Oldlim=\lim
\renewcommand{\int}{\Oldint\limits} 
\renewcommand{\sum}{\Oldsum\limits} 
\renewcommand{\prod}{\Oldprod\limits} 
\renewcommand{\bigcup}{\Oldbigcup\limits} 
\renewcommand{\bigcap}{\Oldbigcap\limits} 
\renewcommand{\lim}{\Oldlim\limits} 
\newcommand{\infint}{\int_{-\infty}^{\infty}}
\newcommand{\iiddist}{\overset{\mathrm{iid}}{\sim}}
\newcommand{\norm}[1]{\left\lVert#1\right\rVert}
\newcommand{\boldred}[1]{\textbf{\textcolor{red}{#1}}}


\begin{document}

In Class: Feb 10

RJ Cass

$X_i \iiddist Pareto(\theta_1, \theta_2): f(x|\theta) = \frac{\theta_1 \theta_2^{\theta_1 - 1}}{x^{\theta_1 + 1}}; x > \theta_2;  \theta_1, \theta_2 > 0$

Find the sufficient statistic for $\theta = (\theta_1, \theta_2)$.

$$
\begin{aligned}
L(\theta|x) 
&= \prod_{i=1}^n \frac{\theta_1 \theta_2^{\theta_1 - 1}}{x_i^{\theta_1 + 1}}I(x > \theta_2) \\
&= \theta_1^n \theta_2^{n(\theta_1 - 1)}\prod \frac{1}{x_i^{\theta_1 + 1}}I(x > \theta_2)  \\ 
&= \theta_1^n \theta_2^{n(\theta_1 - 1)}e^{ln(\sum x_i^{-(\theta_1 + 1)}I(x > \theta_2))} \\
&= \theta_1^n \theta_2^{n(\theta_1 - 1)}e^{-(\theta_1 + 1)\sum ln(x_i) + I(x > \theta_2)} \\
\end{aligned}
$$

Using what we talked about in class of the $T(x)$ being in the exponent, we have the two pieces: $\sum ln(x_i)$ and $x > \theta_2$. The second term can be simplifed to just needing the minimum of $x$ to be greater than $\theta_2$:

$\therefore T(\theta) = (\sum ln(x_i), x_{(1)})$

\end{document}